\documentclass{beamer}

\usepackage{amsmath}
\usepackage{color}
\usepackage{everysel}
\usepackage{mathrsfs}

\DeclareMathOperator{\conv}{\mathrm{conv}}

% Capa
\title{Coloração de Arestas}
\subtitle{
    Nemhauser, George L. e Sungsoo Park.\\
    ``A polyhedral approach to edge coloring.''\\
    Operations Research Letters 10.6 (1991): 315-322.}
\author{Guilherme Guidotti Brandt (235970)}
\institute{IC - Unicamp}
\date{}

\begin{document}
    \frame{\titlepage}

    \begin{frame}{Problema da coloração de arestas}

        {\bf Problema da coloração de arestas}: Dado um grafo {\color{blue}$G = (V, E)$}, encontrar uma atribuição de cores às arestas de {\color{blue}$G$} com o menor número de cores, de forma que arestas adjacentes tenham cores distintas.

        \pause
        \vspace{.5cm}
        Equivalente: encontrar uma decomposição por emparelhamentos (que vamos chamar de {\color{blue} coloração de arestas}) de {\color{blue}$G$} de cardinalidade mínima.
        
        \vspace{.5cm}
        \pause
        O {\color{blue} índice cromático $\chi^\prime(G)$} é o menor número de cores necessário para colorir as arestas do grafo {\color{blue} $G$}.

        \vspace{.5cm}
        \pause
        Politopo:
        $${\color{blue} P_\mathrm{ColA}(G)} = \conv \{{\color{blue} \chi^C} \colon {\color{blue} C} \text{ é uma {\color{blue} coloração de arestas} de } {\color{blue} G} \}$$
    \end{frame}

    \begin{frame}{Teorema de Vizing}

        {\color{blue} \bf Teorema (Vizing)}. Se $G$ é um grafo simples, então {\color{blue} $\chi^\prime(G) = \Delta(G)$} ou {\color{blue} $\Delta(G) + 1$}.
        \pause

        \vspace{.5cm}
        {\color{blue} {\bf Prova}}. Existe um algoritmo que dá uma coloração com {\color{blue} $\Delta(G) + 1$} cores para qualquer grafo simples  em tempo polinomial {\color{blue} (Ehrenfeucht, Faber \& Kierstead, 1984; Lovasz, Plummer, 1986)}. \qed
        \pause

        \vspace{.5cm}
        Sobra um problema de decisão: decidir se um grafo simples {\color{blue} $G$} pode ser colorido com exatamente {\color{blue} $\Delta(G)$} cores.
        \begin{itemize}
            \item Problema NP-completo mesmo quando restrito a grafos 3-regulares {\color{blue} (Holyer, 1981)}.
        \end{itemize}
    \end{frame}

    \begin{frame}{Formulação com PLI}
        O problema é equivalente a encontrar uma cobertura das arestas por emparelhamentos maximais no grafo:
        {\color{blue}
            \begin{align*}
                (\mathrm{IP})\quad&\chi^\prime(G) = \min \mathbf{1} x\\
                &Ax \geq \mathbf{1}\\
                &x \in \mathbb{Z}_{\geq 0}
            \end{align*}
        }
        Onde {\color{blue} $A$} é a matriz de incidência das arestas e emparelhamentos de {\color{blue} $G$}, i.e., {\color{blue} $a_{ij} = 1$} sse a aresta {\color{blue} $i$} é parte do emparelhamento {\color{blue} $j$}.
        \pause
        A matriz {\color{blue} $A$} tem {\color{blue} $|E|$} linhas, e uma coluna para cada emparelhamento maximal em {\color{blue} $G$}.

        \pause
        \vspace{.25cm}
        Podemos notar que, em qualquer solução ótima, {\color{blue} $x \in \mathbb{B}$}.

        \pause
        \vspace{.25cm}
        A cobertura não precisa ser exata: se alguma aresta é escolhida duas vezes, podemos escolher arbitrariamente um dos emparelhamentos da solução em que ela aparece.
    \end{frame}

    \begin{frame}{Formulação com PLI}
        Outras formulações são possíveis com número polinomial de variáveis. Por exemplo, com variáveis {\color{blue} $x_{e,c} = 1$} sse a aresta {\color{blue} $e$} tem a cor {\color{blue} $c$}. Um problema desse tipo de formulação é a quantidade de simetrias.

        \vspace{.5cm}
        \pause
        A formulação {\color{blue} $(\mathrm{IP})$} é interessante porque a relaxação linear dá uma aproximação boa de {\color{blue} $\chi^\prime(G)$}, e permite explorar geração de colunas na solução do problema.
    \end{frame}

    \begin{frame}{Problema da coloração fracionária de arestas}

        Temos a seguinte relaxação linear de {\color{blue} $(\mathrm{IP})$} e seu dual:
        {\color{blue}
            \begin{align*}
                (\mathrm{LP})\quad&\chi^\prime_\mathrm{LP}(G) = \min \mathbf{1} x
                &(\mathrm{DLP})\quad&z_\mathrm{DLP}(G) = \max \mathbf{1} u\\
                &A x \geq 1&&A^\top u \leq 1\\
                &x \geq 0&&u \geq 0                
            \end{align*}
        }
        Note que o dual corresponde a atribuir pesos não-negativos a arestas em {\color{blue} $G$} de forma que a soma dos pesos em cada emparelhamento não ultrapasse 1, maximizando a soma dos custos das arestas.

        \pause
        \vspace{.5cm}
        {\color{blue} {\bf Proposição}}. {\color{blue} $\chi^\prime_\mathrm{LP}(G) \geq \Delta(G)$}

        \vspace{.5cm}
        {\color{blue} {\bf Prova}}. O conjunto de arestas incidentes sobre um vértice de grau {\color{blue} $\Delta(G)$} em {\color{blue} $G$} dá uma solução viável para o dual com custo {\color{blue} $\Delta(G)$}. \qed
    \end{frame}

    \begin{frame}{Problema da coloração fracionária de arestas}

        {\color{blue} {\bf Proposição}}. Se {\color{blue} $\chi^\prime_\mathrm{LP}(G) > \Delta(G)$}, então {\color{blue} $\chi^\prime(G) = \Delta(G) + 1$}, e se {\color{blue} $\chi^\prime_\mathrm{LP}(G) = \Delta(G)$} e existe solução ótima inteira de {\color{blue} $(\mathrm{LP})$}, então {\color{blue} $\chi^\prime(G) = \Delta(G)$}. \qed
        \pause

        \vspace{.5cm}
        {\color{blue} {\bf Teorema}}. O problema da otimização para o programa linear {\color{blue} $(\mathrm{LP})$} pode ser resolvido em tempo polinomial.

        \pause
        \vspace{.5cm}
        {\color{blue} {\bf Prova}}. Basta mostrar que {\color{blue} $(\mathrm{DLP})$} é separável em tempo polinomial.
        \pause
        O problema da separação para {\color{blue} $(\mathrm{DLP})$} corresponde a, dada uma atribuição de pesos não-negativos para as arestas de {\color{blue} $G$}, encontrar um emparelhamento de peso acima de 1.
        \pause
        Isso pode ser feito em tempo polinomial com um algoritmo de emparelhamento de custo máximo (e.g. algoritmo de Edmonds). \qed

        \pause
        \vspace{.5cm}
        Fica um caso remanescente: {\color{blue} $\chi^\prime_\mathrm{LP}(G) = \Delta(G)$}, mas nenhuma solução ótima conhecida é inteira.
    \end{frame}

    \begin{frame}{Desigualdades válidas}
        Seja {\color{blue} $S \subseteq V$} e {\color{blue} $E(S)$} o conjunto de arestas com ambos os extremos em {\color{blue}$S$}. Um pareamento qualquer pode cobrir no máximo {\color{blue} $\lfloor \frac{1}{2} |U| \rfloor$} arestas de {\color{blue} $E(U)$}.

        \pause
        \vspace{.5cm}
        Isso nos dá a seguinte família de desigualdades válidas para {\color{blue} $P_{ColA}(G)$ (Seymour, 1979; Stahl, 1979)}:
        {\color{blue}
        \begin{align*}
            \sum_{\{j \colon M_j \cap E^\prime \neq \emptyset\}} x_j \geq \left\lceil \dfrac{|E^\prime|}{\lfloor \frac{1}{2} |U| \rfloor} \right\rceil
            &&\forall U \subseteq V, E^\prime \subseteq E(U)
        \end{align*}
        }

        onde a variável {\color{blue} $x_j$} corresponde ao emparelhamento maximal {\color{blue} $M_j$}.
    \end{frame}

    \begin{frame}{Desigualdades válidas}
        Em particular, é interessante pensar no caso em que {\color{blue} $|U| = k$} é ímpar e o subgrafo {\color{blue} $C = (U, E^\prime) \subseteq G$} é um ciclo. Nesse caso, {\color{blue}
        $$\left\lceil \dfrac{|E^\prime|}{\lfloor \frac{1}{2} |U| \rfloor} \right\rceil = \lceil (2k + 1)/k \rceil = 3$$
        }
        \pause
        e temos a família de restrições de ciclos ímpares
        {\color{blue}
        \begin{align*}
            \sum_{\{j \colon M_j \cap C \neq \emptyset\}} x_j \geq 3
            &&\forall \text{ ciclo ímpar } C
        \end{align*}
        }
        \pause
        Essas restrições em geral não são combinação cônica das restrições de {\color{blue} $(\mathrm{LP})$} (exceto quando {\color{blue} $|C| = 3$}). Para alguns grafos essas restrições definem faceta {\color{blue} (Park, 1989)}.
    \end{frame}

    \begin{frame}{Separação para grafos 3-regulares}
        Seja {\color{blue} $(\mathrm{ALP})$} o programa linear obtido adicionando as restrições de cíclos ímpares em {\color{blue} $(\mathrm{LP})$} e {\color{blue} $\chi^\prime_\mathrm{ALP}(G)$} seu valor ótimo.

        \vspace{0.5cm}
        {\color{blue} \bf Lema}. Se {\color{blue} $G$} é 3-regular e {\color{blue} $\chi^\prime_\mathrm{ALP}(G) = 3$}, então todo emparelhamento com peso positivo na solução ótima é perfeito.

        \vspace{0.5cm}
        {\color{blue} \bf Teorema}. Se {\color{blue} $G$} é 3-regular e {\color{blue} $\chi^\prime(G) = 4$}, então {\color{blue} $\chi^\prime_\mathrm{ALP}(G) > 3$}.

        \pause
        \vspace{0.5cm}
        Com isso, no caso em que {\color{blue} $G$} é 3-regular e a solução ótima é facionária com custo 3, podemos resolver o problema da separação com as restrições de ciclo ímpar em tempo polinomial: basta remover um emparelhamento com peso positivo para encontrar uma 3-coloração ou uma restrição de ciclo ímpar violada.
    \end{frame}

    \begin{frame}{Generalização}
        O teorema anterior pode ser generalizado para grafos de qualquer índice cromático:

        \pause
        Seja {\color{blue} $\mathscr{G}_{k-1}$} a família de grafos com {\color{blue} $\Delta = k - 1$} e {\color{blue} $\chi^\prime$ = k}.
        
        \pause
        Para um grafo {\color{blue} $G$} com {\color{blue} $\Delta = k$}, as desigualdades de ciclos ímpares podem ser generalizadas para:
        {\color{blue}
        \begin{align*}
            \sum_{\{j \colon M_j \cap H \neq \emptyset\}} x_j \geq k
            &&\forall H \subseteq G, H \in \mathscr{G}_{k-1}
        \end{align*}
        }
        \pause
        Defina {\color{blue} $(\mathrm{ALP}_k)$} como o problema linear obtido adicionando essas restrições a {\color{blue} $(\mathrm{LP})$}. Temos o seguinte teorema:

        \vspace{0.5cm}
        {\color{blue} \bf Teorema}. Se {\color{blue} $G$} é k-regular e {\color{blue} $\chi^\prime(G) = k + 1$}, então {\color{blue} $\chi^\prime_\mathrm{ALP_k}(G) > k$}.

        \pause
        \vspace{0.5cm}
        Infelizmente, a separação nesse caso parece ser bem mais difícil.
    \end{frame}

    \begin{frame}{Geração de colunas}

        A formulação {\color{blue} $LP$} tem um número de variáveis exponencial no tamanho do grafo. Seria inviável resolver normalmente.

        \pause
        \vspace{0.5cm}
        Para contornar isso, podemos executar o Simplex olhando apenas para a base, e ``gerar'' colunas com custo reduzido negativo conforme necessário.

        \pause
        \vspace{0.5cm}
        O custo reduzido da variável (coluna) {\color{blue} $j$} é dado por
        {\color{blue} $$r_j = c_j - u^\top A_{*j}$$}
        Onde {\color{blue} $c$} é a direção de otimização, {\color{blue} $u$} é o vetor das variáveis duais e {\color{blue} $A$} é a matriz que define o poliedro.
    \end{frame}

    \begin{frame}{Geração de colunas}

        Podemos encontrar uma coluna com custo reduzido negativo resolvendo o seguinte problema de otimização:
        {\color{blue}
            \begin{align*}
                (\mathrm{PR})\quad&z(u) = \max u^T a_j\\
                &a_j \in A
            \end{align*}
        }
        Isto é, encontrar uma coluna de {\color{blue} $A$} cujo produto interno com {\color{blue} $u^T$} seja máximo (minimizando o custo reduzido).
        
        \pause
        \vspace{0.5cm}
        Se {\color{blue} $z(u) > c_j$}, a coluna {\color{blue} $j$} entra na base; se não, não existe coluna com custo reduzido negativo, e o Simplex termina.
    \end{frame}

    \begin{frame}{Geração de colunas para (LP)}
        
        Temos o seguinte programa linear:
        {\color{blue}
        \begin{align*}
            (\mathrm{LP})\quad&\chi^\prime_\mathrm{LP}(G) = \min \mathbf{1} x\\
            &A x \geq 1\\
            &x \geq 0
        \end{align*}}
        \pause
        Que nos dá o seguinte problema de precificação:
        {\color{blue}
        \begin{align*}
            (\mathrm{PR})\quad&z(w) = \max w^T y\\
            &\sum_{e \in \delta(v)} y_e \leq 1 &\forall v \in V\\
            &y_e \in \mathbb{B}&\forall e \in E
        \end{align*}}
        \pause
        Que é equivalente a emparelhamento de custo máximo (onde os custos são dados por {\color{blue} $w$}, o vetor das variáveis duais no problema mestre).
    \end{frame}

    \begin{frame}{Geração de colunas para (ALP)}

        Temos o seguinte programa linear:
        {\color{blue}
        \begin{align*}
            (\mathrm{ALP})\quad&\chi^\prime_\mathrm{LP}(G) = \min \mathbf{1} x\\
            &A x \geq 1\\
            &\sum_{\{j \colon M_j \cap C \neq \emptyset\}} x_j \geq 3
            &\forall \text{ ciclo ímpar } C\\
            &x \geq 0
        \end{align*}}
    \end{frame}

    \begin{frame}{Geração de colunas para (ALP)}

        Que nos dá o seguinte problema de precificação:
        {\color{blue}
        \begin{align*}
            (\mathrm{PRA})\quad&z(w, u) = \max w^T y + u^\top \pi\\
            &\sum_{e \in \delta(v)} y_e \leq 1 &\forall v \in V\\
            &\pi_C \leq \sum_{e \in C} y_e&\forall\text{ ciclo ímpar }C\\
            \uncover<2->{&\sum_{e \in E(S)} y_e \leq \frac{1}{2}(|S| - 1) &\forall S \subseteq V, |S|\text{ ímpar}}\\
            &\pi_C \in \mathbb{B}&\forall\text{ ciclo ímpar }C\\
            &y_e \in \mathbb{B}&\forall e \in E
        \end{align*}}
        Onde {\color{blue} $w$} é o vetor de variáveis duais das restrições de aresta e {\color{blue} $u$} das restrições de ciclos ímpares.
    \end{frame}

    \begin{frame}{Algoritmo para grafos 3-regulares}

        \begin{enumerate}
            \item Inicialize o Simplex com um conjunto razoável de colunas que permita uma solução
            \begin{itemize}
                \item Isso pode ser feito usando o Teorema de Vizing e/ou soluções heurísticas.
            \end{itemize}
            \item Resolva a relaxação linear do problema.
            \item Se o valor ótimo é 3:
            \begin{enumerate}
                \item Se a solução é inteira ou nenhuma restrição de ciclo ímpar foi violada, termina com {\color{blue} $\chi^\prime(G) = 3$}.
                \item Se a solução não é inteira e alguma restrição de ciclo ímpar foi violada, adiciona ao sistema e volta ao passo {\color{blue} 2}.
            \end{enumerate}
            \item Se o valor ótimo é maior que 3, gere uma coluna com o problema de precificação.
            \begin{enumerate}
                \item Se o preço da solução ótima for maior que 1, adicione a coluna e volte ao passo {\color{blue} 2}.
                \item Se não, não existe coluna com preço reduzido negativo, e termina com {\color{blue} $\chi^\prime(G) = 4$}.
            \end{enumerate}
        \end{enumerate}
    \end{frame}


    \begin{frame}{Resultados computacionais}

        \begin{itemize}
            \item Algoritmo implementado em FORTRAN, executado em um IBM 9375, com método simplex para programação linear e branch-and-bound para programação inteira.
            
            \item Testado com instâncias aleatórias e alguns snarks (grafos 3-regulares com índice cromático 4 e giro $\geq 5$).
            
            \begin{itemize}
                \item Os grafos aleatórios gerados eram todos 3-coloríveis, o que que é esperado: quase todos os grafos têm índice cromático igual ao seu grau máximo {\color{blue} (Erdős e Wilson, 1977)}.
                \item Em geral, o algoritmo não encontra muitas soluções inteiras: os cortes não descrevem muito precisamente o poliedro para grafos 3-regulares e 3-coloríveis.
                \item Em geral, vários cortes são necessários para mostrar que os snarks precisam de 4 cores.
                \item O tempo de execução geralmente é dominado pela geração de colunas.
            \end{itemize}
        \end{itemize}
    \end{frame}

\end{document}
